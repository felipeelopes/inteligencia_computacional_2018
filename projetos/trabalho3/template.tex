% =======================================================================
% =                                                                     =
% =======================================================================
% -----------------------------------------------------------------------
% - Author:     Chaua Queirolo                                          -
% - Version:    001                                                     -
% -----------------------------------------------------------------------
\documentclass[a4paper,11pt]{article}    

% =======================================================================
% PACKAGES
% =======================================================================

% Language support
\usepackage[brazil]{babel}
\usepackage[utf8]{inputenc}
\usepackage[T1]{fontenc}
\usepackage{ae,aecompl}

% Configuration
\usepackage{url}
\usepackage{enumerate}
\usepackage{color}
\usepackage[svgnames,table]{xcolor}
\usepackage[margin=2cm,includefoot]{geometry}

% Tabular
\usepackage{multirow}
\usepackage{multicol}

% Images
\usepackage{graphicx}
\usepackage[scriptsize]{subfigure}
\usepackage{epstopdf}
\usepackage{float}% http://ctan.org/pkg/{multicol,lipsum,graphicx,float}

% Math
\usepackage{mdwtab}	% bug rowcolor
\usepackage{amssymb}
\usepackage{amsmath}
\usepackage{footnote}

% References
\usepackage[sort,nocompress]{cite}

% =======================================================================
% VARIABLES
% =======================================================================

% Space between the lines in a table
\renewcommand{\arraystretch}{1.3}

% Define a new column type
\newcolumntype{x}[1]{>{\raggedright\hspace{0pt}}p{#1}}%

% Controle das Margens
\sloppy
\tolerance=9999999

% Espaço entre colunas
\setlength{\columnsep}{.9cm}


% Configuration
\usepackage{lipsum}
\usepackage{blindtext}

% =======================================================================
% HEADER
% =======================================================================

\title{Resenha: Aplicação de redes neurais na análise e na
 concessão de crédito ao consumidor}
\author{Felipe Eduardo Lopes\\E-mail: {\tt felipe\_lopes@outlook.com}}
\date{}

\newenvironment{Figure}
  {\par\medskip\noindent\minipage{\linewidth}}
    {\endminipage\par\medskip}

% =======================================================================
% DOCUMENT
% =======================================================================
\begin{document}

\maketitle

\begin{multicols}{2}

\section{Objetivo}
O cérebro humano é um computador altamente complexo, não-linear e paralelo. O esforço científico para reproduzir seu funcionamento de forma computacional deu origem as Redes Neurais Artificiais, as quais são modelos baseados no comportamento do mesmo. Uma Rede Neural Artificial é um processador maciçamente paralelo distribuído, construído de unidades de processamento simples, que têm a propensão natural para armazenar conhecimento experimental e torná-lo disponível para uso. \cite{ref:haykin1}
O Artigo de 2007 \cite{ref:art2007} tem como objetivo a utilização de Redes Neurais Artificiais aplicada à Análise de Crédito com o foco de identificar bons e maus pagadores.


\section{Métodos}
Os autores do trabalho (2007)\cite{ref:art2007} utilizam Redes Neurais Artificias baseadas em Multilayer Perceptron (MLP), aplicando algoritmo de backpropagation.

\subsection{Resultados Objtidos}
O algoritmo de backpropagation foi aplicado em uma amostra aleatória de 2.475 clientes que realizaram compras a prazo
da base de dados de uma rede varejista. Foi dividindo a amostra para processamento e predição, a melhor rede propiciou 79\%, 71\% e 85\% de acertos sobre o perfil de pagamento em cada uma das
fases de treinamento, validação e teste,  respectivamente. Tendo em vista que foi levado em consideração na pesquisa apenas número reduzido de variáveis de cadastro, os resultados sugerem que as redes neurais podem representar uma promissora técnica para a análise de concessão de crédito ao consumidor.

\bibliographystyle{plain}
\bibliography{referencias}

\end{multicols}
\end{document}

